\section{Estructura de los modelos matemáticos}
\label{sec:formulations}

\begin{frame}{Estructura De Un Modelo Matemático}

Todos los modelos de IO, incluido el de Programación Lineal (PL), constan de tres componentes básicos.

  \begin{enumerate} \justifying 
  \item Los \alert{parámetros} o variables externas que no están bajo nuestro control.
  \item Las \alert{variables de decisión} que pretendemos determinar.
  \item El \alert{objetivo} (la meta) que necesitamos optimizar (maximizar o minimizar).
  \item Las \alert{restricciones} que la solución debe satisfacer.
  \end{enumerate}
   
\end{frame}

\input{examples/example_reddy-mikks_taha}
\input{examples/example_allocation_gupta.tex}
\input{examples/example_diet_gupta}
\input{examples/example_blending_gupta}
\input{examples/example_media-selection_gupta}
\begin{frameExample}{Inspección}{}
  % EXAMPLE 2.6-5 {lnspection Problem} Gupta
Una empresa tiene dos grados de inspectores, I y II para llevar a cabo la inspección de control de calidad. Se deben inspeccionar al menos 1,500 piezas en un día de 8 horas. El inspector de grado I puede verificar 20 piezas en una hora con una precisión del 96\%. Grado II el inspector verifica 14 piezas por hora con un precisión del 92\%. Los salarios del inspector de grado I son \$ 5 por hora, mientras que los de grado II inspector son \$ 4 por hora. Cualquier error cometido por un inspector cuesta \$ 3 a la empresa. Si hay, en total, 10 grado I
inspectores y 15 inspectores de grado II en la empresa, encuentre la asignación óptima de inspectores que minimiza el costo diario de inspección.
    
\end{frameExample}



%%% Local Variables:
%%% mode: latex
%%% TeX-master: "../slides"
%%% End:

\begin{frameExample}{Mezcla de Productos}{}
  % EXAMPLE 2.6-6 (Product Mix Problem) Gupta
Una compañía química produce dos productos, $X$ e $Y$. Cada unidad de producto $X$ requiere 3 horas en operación I y 4 horas en operación II, mientras que cada unidad de producto $Y$ requiere 4 horas en operación I y 5 horas en operación II. El tiempo total disponible para las operaciones I y II es 20 horas y 26 horas respectivamente. La producción de cada unidad de producto $Y$ también da como resultado dos unidades de un subproducto $Z$ sin costo adicional. El producto $X$ se vende con una ganancia de \$ 10 / unidad, mientras que $Y$ se vende con una ganancia de \$ 20 / unidad. El subproducto $Z$ aporta un beneficio unitario de \$ 6 si se vende; en caso de que no se pueda vender, el costo de destrucción es de \$ 4 / unidad. Los pronósticos indican que no se pueden vender más de 5 unidades de $Z$. Formule el modelo L.P. para determinar las cantidades de $X$ e $Y$ que se producirán, teniendo en cuenta $Z$, de modo que la ganancia obtenida sea máxima.
    
\end{frameExample}



%%% Local Variables:
%%% mode: latex
%%% TeX-master: "../slides"
%%% End:

\begin{frameExample}{Mezcla de Productos (Fracciones)}{}
  % EXAMPLE 2.6-7 (Product Mix Problem) Gupta
Una empresa fabrica tres productos A, B y C. El tiempo para fabricar el producto A es el doble que para B y tres veces para C y si toda la mano de obra se dedica a la fabricación del producto A, se pueden producir 1.600 unidades de este producto. Estos productos deben producirse en una proporción de 3: 4: 5. Hay demanda de al menos 300, 250 y 200 unidades de productos A, B y C y el beneficio obtenido por unidad es de \$ 90, \$ 40 y \$ 30 respectivamente. Formule el problema como un problema de programación lineal.

{\centering
\includegraphics[scale=0.5]{example_product-mix02_gupta}
\par}

\end{frameExample}



%%% Local Variables:
%%% mode: latex
%%% TeX-master: "../slides"
%%% End:

\begin{frameExample}{Corte de Papel}{}
  % EXAMPLE 2.6-8 (Trim Loss Problem} 
  \only<1>{%
Una fábrica de papel produce rollos de papel utilizados para hacer cajas registradoras. Cada rollo de papel tiene una longitud de 100 m y se puede utilizar en anchos de 3, 4, 6 y 10 cm. El proceso de producción de la compañía da como resultado rollos de 24 cm de ancho. Por lo tanto, la empresa debe cortar su rollo de 24 cm al ancho deseado. Tiene seis alternativas básicas de corte de la siguiente manera:%
  }

{\centering
\includegraphics<1>[scale=0.5]{example_trim-loss_gupta}
\par}

\only<2>{%
  La demanda mínima para los cuatro rollos es la siguiente:
  
{\centering
  \begin{tabular}{cr}
    \toprule
    Ancho del rollo (cm) & Demanda\\
     \midrule
    2 & 2,000 \\
    4 &3,600 \\
    6 &1,600 \\
    10 &500\\
    \bottomrule
  \end{tabular}
  \par}

La fábrica de papel desea minimizar el desperdicio resultante del recorte al tamaño. Formule el modelo L.P.%
}

\end{frameExample}



%%% Local Variables:
%%% mode: latex
%%% TeX-master: "../slides"
%%% End:

\begin{frameExample}{Planeación de la Producción}{}
  % EXAMPLE 2.6-9 (Production Planning Problem) 
Una fábrica elabora un producto cuya unidad consta de 5 unidades de la parte A y 4 unidades de la parte B. Las dos partes A y B requieren diferentes materias primas, de las cuales están disponibles 120 unidades y 240 unidades respectivamente. Estas piezas pueden fabricarse por tres métodos diferentes. Los requisitos de materia prima por producción y el número de unidades para cada parte producida se detallan a continuación. Formule el modelo L.P. para determinar el número de corridas de producción para cada método a fin de maximizar el número total de unidades completas del producto final.

{\centering
\includegraphics[scale=0.5]{example_production-planning_gupta}
\par}


\end{frameExample}



%%% Local Variables:
%%% mode: latex
%%% TeX-master: "../slides"
%%% End:



%%% Local Variables:
%%% mode: latex
%%% TeX-master: "slides"
%%% End:
