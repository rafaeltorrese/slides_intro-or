\begin{frameExample}{Mezcla de Productos (Fracciones)}{}
  % EXAMPLE 2.6-7 (Product Mix Problem) Gupta
Una empresa fabrica tres productos A, B y C. El tiempo para fabricar el producto A es el doble que para B y tres veces para C y si toda la mano de obra se dedica a la fabricación del producto A, se pueden producir 1.600 unidades de este producto. Estos productos deben producirse en una proporción de 3: 4: 5. Hay demanda de al menos 300, 250 y 200 unidades de productos A, B y C y el beneficio obtenido por unidad es de \$ 90, \$ 40 y \$ 30 respectivamente. Formule el problema como un problema de programación lineal.

{\centering
\includegraphics[scale=0.5]{example_product-mix02_gupta}
\par}

\end{frameExample}



%%% Local Variables:
%%% mode: latex
%%% TeX-master: "../slides"
%%% End:
